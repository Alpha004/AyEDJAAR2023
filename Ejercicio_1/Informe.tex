\documentclass{article} % This command is used to set the type of document you are working on such as an article, book, or presenation

% \usepackage{geometry} % This package allows the editing of the page layout
% \usepackage{amsmath}  % This package allows the use of a large range of mathematical formula, commands, and symbols
% \usepackage{graphicx}  % This package allows the importing of images
\usepackage[T1]{fontenc}
\usepackage{blindtext}
\usepackage[]{algorithm2e}
% \usepackage[T1]{fontenc}
\title{Comparativa de Algoritmos de Ordenamiento}
\date{2023\\ Julio}
\author{Luis Borit Guitton - Abel Borit Guitton - Jesus Alpaca Rendon\\ Maestria en Ciencias de la Computacion, Universidad Nacional de San Agustin}     
\begin{document}
\maketitle
\section{Introduccion}
El presente informe muestra la comparativa de tiempos de ejecucion realizada entre 5 algoritmos de ordenamiento como son: Binary Insertion Sort, Bubble sort, Quick sort, Selection sort y Merge sort.
Implementados en 3 diferentes lenguajes de programacion: Python, Golang y C++. Los algoritmos fueron sometidos a pruebas para obtener un tiempo promedio de ejecucion y ver los resultados
en una grafica comparativa
\section{Algoritmos}
Los algoritmos seleccionados para el presente trabajo de investigacion son los siguientes:

* Binary Tree

* Bubble sort

* Quick sort

* Selection sort

* Merge Sort

\begin{enumerate}
    \item Binary Insertion Sort
    \item Bubble Sort
    
          Este algoritmo realiza el ordenamiento o reordenamiento de una lista a de n valores, en este caso de n términos numerados del 0 al n-1; consta de dos bucles anidados, uno con el índice i, que da un tamaño menor al recorrido de la burbuja en sentido inverso de 2 a n, y un segundo bucle con el índice j,
          con un recorrido desde 0 hasta n-i, para cada iteración del primer bucle, que indica el lugar de la burbuja. La burbuja son dos términos de la lista seguidos, j y j+1, que se comparan: si el primero es mayor que el segundo sus valores se intercambian.

          \begin{algorithm}[H]
              \KwData{$a_{1}$,$a_{2}$,$a_{3}$,$a_{4}$...$a_{(n-1)}$}
              \KwResult{ordered list}
              initialization\;
              \For{$i$ \KwTo n-1}{
                  \For{$j$ \KwTo n-i-1}{
                      \If{$a_{j}$ > $a_{j+1}$}{
                          go to next section\;
                          $aux\gets a_{j}$\;
                          $a_{j}\gets a_{j+1}$\;
                          $a_{j+1}\gets aux$\;                          
                      }
                  }
              }              
              \caption{Bubble Algorithm}
          \end{algorithm}

    \item Quick Sort
    \item Selection Sort
    \item Merge Sort
\end{enumerate}
\section{Implementacion}
\section{Resultados}
\section{Conclusiones}
\end{document}